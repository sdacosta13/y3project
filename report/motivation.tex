\label{motivations}
\subsection{Motivations}
The motivation for this project stems from previous courses I have taken such as `Fundamentals of Computer Architecture', `Microcontrollers' and `Operating Systems'. When taking these course I really enjoyed the challenges behind working within an ARM based environment, such as working with few `variables' and having no pre-made software available to you. For me, these challenges raised the question of how viable it is to write an operating system for a microcontroller.
While I had done a simple form of this for my Microcontrollers course, I wanted to take it further by implementing more complex features. In addition to this I wanted to improve upon the work I had done. This work had been relatively rushed and messy as I was having to learn on the go, and I did not have much time to re-factor. From this I derived two main goals for this project; I wanted to develop an OS which was better structured, and I wanted to develop some sort of process management service for the ARM chip. The operating systems course should help with the development of the process management service, as the notes I have explain how operating systems manage processes and threads within an OS. This project is more concerned with developing threads rather than processes, the distinction being that a thread is usually a segment of a program running as a process. 

Finding ways of keeping my work organised became a large part of this project for me. Most of my programming experience has, until now, been focused on higher level languages such as C\# and Java. Due to this, developing for a low level language felt quite jarring due to the following characteristics of ARM, program structure can become disorganised and hard to read without self-imposed discipline. Braces and indentation are not enforced, which would usually expose the control flow of the program. Automated type checking does not exist. These problems reinforced the necessity of commenting in all my code, even beyond ARM. Consistently commenting an a specific style also became a key strategy to ensuring my code was readable. I often found myself trading off legibility against efficiency.

\subsection{Background}
The system I built has it roots in the Microcontroller's course, and it derives much of its environment from the work done there. The system was built for the graphical debugger named `Komodo' \cite{kmd}. This acts as a `front end' for `back end' processor models. In my case the model I used was  and emulator called `Jimulator'. It also provides me with the assembler for my ARM code, as well as the ability to load and run the assembled code on a virtual ARM processor. The debugging facilities it provides allow me to pause the code at breakpoints I set and inspect the state of the processor's memory and registers. My working environment also consists of a few plug-ins for Jimulator. These plug-ins are as follows:

\begin{itemize}
	\item An 8-bit clock-based timer exposed as a byte in memory.
	\item 3 x 4 virtual keypad with an interface set-up in the memory.
	\item A 320px x 240px virtual RGB display with an interface set-up in the memory of the ARM chip.
\end{itemize}

These plug-ins for the Jimulator emulator expose themselves at static memory addresses with various interfaces.
The timer appears as a read-only memory location at 0xF100\_1010. This timer is free-running and increments at 1kHz. 
The keypad simulates a matrix keypad. It works by allowing the ARM processor to activate scan lines and read the resulting input to determine which keys have been pressed. It also appears as a single byte containing both the scan-line bits and the resulting output bits.

\begin{figure}[ht!]
	\includegraphics[width=0.5\linewidth]{figures/keypad.png}\centering
	\caption{Keypad Matrix Setup}
	\label{fig:keypad}
\end{figure}

The virtual screen is made of a `vscreen' program, connected to the Jimulator via a plug-in. The plug-in exposes the access to the screen via a frame buffer set-up in memory starting at 0xAC00\_0000. The memory set-up consists of 3 bytes for each pixel running left to right, top to bottom. The 3 bytes represent the RGB values of the corresponding pixel.

\begin{figure}[ht!]
	\includegraphics[width=\linewidth]{figures/LCD.png}\centering
	\caption{Frame buffer memory set-up}
	\label{fig:LCDMem}
\end{figure}

The system I created required me to modify this starting environment by replacing the keypad with an new plug-in, which created and handled a new virtual keyboard as detailed later in \autoref{chap:virtualKeyboard}. 

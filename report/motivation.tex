\label{motivations}
The motivation for this project stems from previous courses I have taken such as COMP15111, Fundamentals of Computer Architecture, COMP22712 Micro-controllers, and COMP15212 Operating Systems. When taking these course I really enjoyed the challenges behind working within an ARM based environment, such as working with few 'variables' and having no pre-made software made for you. For me, these challenges raised the question of how viable it is to write an operating system for a microcontroller.
While I had done a simple form of this for COMP22712, I wanted to take it further by implementing more complex features. In addition to this I wanted to improve upon the work I had done in COMP22712. This work hand been relatively rushed and messy as I was having to learn on the go, and I did not have much time to re-factor. From this I derived two main goals for this project. I wanted to develop and OS which was easier to read through and keep organised, and I wanted to develop some sort of process management service for the ARM chip. The operating systems course should help with the development of the process management service should help, as the notes I have detail how operating systems manage processes and threads within an OS. This project is more concerned with developing threads rather than processes, the distinction being that a thread is usually a segment of a program running as a process. 

Finding ways of keeping my work organised became a large part of this project for me. Arm code is very hard to keep organised as it doesn't have visual aids such as braces to expose the control flow of the program. This can make it very hard to read and edit. In addition, you can only name memory locations. Not being able to name registers means that you have to keep track of what is where at all times. All of these problems reinforced the necessity of commenting in all my code, even beyond ARM. When I wrote the OS, I had to keep my commenting style as consistent as I could, and name variables appropriately. Very often I found myself trading off readability against efficiency 
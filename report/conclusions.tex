As a learning exercise, this project was quite successful and enjoyable to complete. Not only did I develop a basic operating system, I learnt about some of the challenges faced by low level software developers. In addition, I feel that this project has reinforced good lessons regarding commenting, code consistency, and organisation. I have a new appreciation for documentation and the information that it provides. In this chapter I will discuss where I could improve and where I felt I did well. 


\subsection{Evaluation of Project Strategy.}
This section examines the methods I employed to develop the system effectively. I feel that self-imposing conventions upon myself greatly improved the readability of my code. The most effective method I found to make development more streamlined was organising the structure of my program. Ensuring that my file had a separation of concerns in place meant that when I needed to reference previous code or method, the information I needed was easier to find. This was further aided by the consistent commenting I employed on procedure call headers. The convention to ensure all constants were labelled rather than directly  referenced was definitely helpful, but I feel that I employed the use of this method too late, and as such I did not receive the full benefit from it. 


\subsection{Project Goals}
\subsubsection{Objective 1}
The goal to support the basic functions of an Operating System was achieved markedly. My system provides access to abstract functions used to interact with input and output devices. It can also initialise the system to a safe valid state. This accomplishes the main points in Objective 1, but goes no further. With more work I could add more functions such as access to drawing functions for the virtual screen, or I could develop dynamic data structures for the user. I also would have liked to develop some basic memory exception handling mechanisms
\subsubsection{Objective 2}
The goal to develop a virtual keyboard went well, but I feel a more robust implementation would be possible with more time. Modifying the virtual environment was a significant technical challenge which required me to experiment with various technologies that I had not used before. Developing the plug-in for Jimulator was quite an achievement considering that I could not find much documentation on how to perform this task. If I had the chance to go back and improve this component, I would like to add a keyboard buffer for to transfer the data from the plug-in. The current solution works, but is somewhat unrealistic when compared to a modern keyboard. 
\subsubsection{Objective 3}
The third goal was to develop the thread management system. I was proud of accomplishing this goal. My implementation does include some glaring flaws, but overcoming this technical challenge was fulfilling. Threading is a task which I have found I can struggle with when developing higher level code, so being able to implement threading at a lower level feels reassuring. My final system does need a rework, to account for the shortfalls regarding multiple instances of the same object code, however I am pleased with the overall result. 






